\documentclass[t, aspectratio=169]{beamer}
\usepackage[utf8]{vietnam}
\usepackage{amsthm,amsmath,amssymb}
\usepackage{lmodern}
\usepackage{graphicx}
\usepackage{hyperref}

% --- CẤU HÌNH LỀ ---
\def\slideContentLeftMargin{0.5cm}
\def\slideContentRightMargin{1cm}
\def\hustHeaderHeight{1cm}
\def\slideContentBotMargin{3.5cm}

\usepackage[blue, 169]{beamerthemeHUST} 

% Căn lề title page
\renewcommand{\titlePageTitleX}{-0.5cm}
\renewcommand{\titlePageTitleY}{-0.5cm}
\renewcommand{\hustTitleAlign}{\alignLeft} 
\renewcommand{\hustSubtitleAlign}{\alignLeft}
\renewcommand{\hustAuthorAlign}{\alignLeft}
\renewcommand{\hustInstAlign}{\alignLeft}

% Thông tin đồ án
\title{HỆ THỐNG QUẢN LÝ KHÁCH SẠN}
\subtitle{Đồ án Lập trình Hướng đối tượng - Nhóm 204}
\author{\texorpdfstring{Trần Đình Khánh - 20237449 \\ Nguyễn Hữu Linh - 20237455}{Trần Đình Khánh - 20237449, Nguyễn Hữu Linh - 20237455}} 


\begin{document}

\hustintropage{}
\hustsectionpage{}

% Trang bìa
\husttitlepage

\begin{frame}{Nội dung trình bày}
	\tableofcontents[hideallsubsections]
\end{frame}

\resetPageCount
\showPageNumber
\useStyleHustFull
\useThinBar 

% =================================================================
% PHẦN 1: GIỚI THIỆU
% =================================================================
\section{Giới thiệu bài toán}

\begin{frame}{1.1 Bối cảnh và bài toán}
	\textbf{Bối cảnh thực tế:}
	\begin{itemize}
		\item Ngành khách sạn cần hệ thống phần mềm quản lý \textbf{chuyên nghiệp}
		\item Quản lý thủ công gây \textbf{sai sót}, \textbf{chậm trễ} và khó mở rộng
		\item Yêu cầu: Đặt phòng, Check-in/out, Hóa đơn, Báo cáo doanh thu
	\end{itemize}
	
	\vspace{0.5em}
	
	\textbf{Mục tiêu dự án:}
	\begin{itemize}
		\item Xây dựng hệ thống \textbf{CRUD} hoàn chỉnh cho Room, Booking, Customer, Invoice
		\item Áp dụng \textbf{4 tính chất OOP}: Encapsulation, Inheritance, Polymorphism, Abstraction
		\item Công nghệ: \textbf{Java 21}, Swing + FlatLaf, JSON Storage, Maven
	\end{itemize}
\end{frame}

\begin{frame}{1.2 Phạm vi hệ thống}
	\begin{columns}[T]
		\begin{column}{0.45\textwidth}
			\textbf{6 Module chức năng:}
			\begin{enumerate}
				\item Quản lý Phòng
				\item Quản lý Đặt phòng
				\item Quản lý Khách hàng
				\item Quản lý Hóa đơn
				\item Đăng nhập/Phân quyền
				\item Báo cáo Thống kê
			\end{enumerate}
			
			\vspace{1em}
			\textbf{Công nghệ:}
			\begin{itemize}
				\item Java 21, Swing + FlatLaf
				\item JSON Storage (Gson)
				\item Maven Build
			\end{itemize}
		\end{column}
		
		\begin{column}{0.55\textwidth}
			\begin{figure}
				\centering
				\includegraphics[width=0.95\textwidth, height=0.6\textheight, keepaspectratio]{PhamViHeThong.drawio.png}
				\caption{Sơ đồ phạm vi hệ thống}
			\end{figure}
		\end{column}
	\end{columns}
\end{frame}

% =================================================================
% PHẦN 2: PHÂN TÍCH YÊU CẦU
% =================================================================
\section{Phân tích yêu cầu}

\begin{frame}{2.1 Đặc tả Actor - 3 Role nghiệp vụ}
	\begin{block}{Hệ thống sử dụng 3 Role nghiệp vụ độc lập (không có kế thừa Actor)}
		\begin{tabular}{|l|l|l|}
			\hline
			\textbf{Actor} & \textbf{Role trong code} & \textbf{Chức năng chính} \\
			\hline
			Quản lý (Manager) & \texttt{MANAGER} & Quản lý phòng, xem báo cáo \\
			\hline
			Lễ tân (Staff) & \texttt{STAFF} & Đặt phòng, check-in/out, hóa đơn \\
			\hline
			Bộ phận Dịch vụ (Service) & \texttt{SERVICE} & Cung cấp dịch vụ khách hàng \\
			\hline
		\end{tabular}
	\end{block}
	
	\vspace{0.5em}
	
	\textbf{Tại sao không dùng kế thừa Actor?}
	\begin{itemize}
		\item Hệ thống dùng \textbf{Permission-based Access Control}
		\item Mỗi Role có tập quyền riêng biệt trong \texttt{PermissionManager.java}
		\item Linh hoạt hơn: Dễ thêm/bớt quyền mà không cần sửa cấu trúc
	\end{itemize}
\end{frame}

\begin{frame}{2.2 Use Case tổng quan}
	\begin{figure}
		\centering
		\includegraphics[width=0.7\textwidth, height=0.7\textheight, keepaspectratio]{TongQuan.drawio.png}
		\caption{Use Case Diagram tổng quan - 3 Actor nghiệp vụ}
	\end{figure}
\end{frame}

\begin{frame}{2.3 Use Case chi tiết - Lễ tân (Staff)}
	\begin{columns}[T]
		\begin{column}{0.5\textwidth}
			\begin{figure}
				\centering
				\includegraphics[width=0.95\textwidth, height=0.55\textheight, keepaspectratio]{LeTan.drawio.png}
				\caption{Use Case Lễ tân}
			\end{figure}
		\end{column}
		
		\begin{column}{0.5\textwidth}
			\textbf{6 Use Case chính:}
			\begin{enumerate}
				\item \textbf{Đăng nhập}: Xác thực qua AuthService
				\item \textbf{Tìm phòng trống}: Lọc theo ngày
				\item \textbf{Đặt phòng mới}: Tạo Booking
				\item \textbf{Check-in}: Đổi status CHECKED\_IN
				\item \textbf{Check-out}: Đổi status COMPLETED
				\item \textbf{Tạo hóa đơn}: Sinh Invoice
			\end{enumerate}
			
			\vspace{0.5em}
			\textbf{Quan hệ \texttt{<<include>>}:}
			\begin{itemize}
				\item ``Đặt phòng'' $\rightarrow$ ``Tìm phòng trống''
				\item ``Check-out'' $\rightarrow$ ``Tạo hóa đơn''
			\end{itemize}
			\small\textit{Include = bắt buộc phải thực hiện}
		\end{column}
	\end{columns}
\end{frame}

\begin{frame}{2.4 Use Case chi tiết - Quản lý (Manager)}
	\begin{columns}[T]
		\begin{column}{0.5\textwidth}
			\begin{figure}
				\centering
				\includegraphics[width=0.95\textwidth, height=0.55\textheight, keepaspectratio]{QuanLy.drawio.png}
				\caption{Use Case Quản lý}
			\end{figure}
		\end{column}
		
		\begin{column}{0.5\textwidth}
			\textbf{6 Use Case chính:}
			\begin{enumerate}
				\item \textbf{Đăng nhập}: Xác thực quyền MANAGER
				\item \textbf{Thêm/Sửa/Xóa phòng}: CRUD Room
				\item \textbf{Xem báo cáo doanh thu}: Dashboard
				\item \textbf{Xem thống kê phòng}: Biểu đồ
			\end{enumerate}
			
			\vspace{0.5em}
			\textbf{Quyền đặc biệt (PermissionManager):}
			\begin{itemize}
				\item \texttt{MANAGE\_ROOMS} - CRUD phòng
				\item \texttt{VIEW\_REPORTS} - Xem báo cáo
				\item \texttt{EXPORT\_REPORTS} - Xuất báo cáo
				\item \texttt{MANAGE\_SERVICES} - Quản lý DV
			\end{itemize}
			
			\vspace{0.5em}
			\small\textit{Manager có gần như full quyền, chỉ thiếu MANAGE\_ACCOUNTS}
		\end{column}
	\end{columns}
\end{frame}

\begin{frame}{2.5 Use Case chi tiết - Bộ phận Dịch vụ (Service)}
	\begin{columns}[T]
		\begin{column}{0.5\textwidth}
			\begin{figure}
				\centering
				\includegraphics[width=0.95\textwidth, height=0.55\textheight, keepaspectratio]{DichVu.drawio.png}
				\caption{Use Case Bộ phận Dịch vụ}
			\end{figure}
		\end{column}
		
		\begin{column}{0.5\textwidth}
			\textbf{4 Use Case chính:}
			\begin{enumerate}
				\item \textbf{Đăng nhập}: Xác thực quyền SERVICE
				\item \textbf{Xem phòng}: Kiểm tra trạng thái
				\item \textbf{Xem dịch vụ}: Danh sách dịch vụ
				\item \textbf{Cung cấp dịch vụ}: Phục vụ khách
			\end{enumerate}
			
			\vspace{0.5em}
			\textbf{Quyền hạn (PermissionManager):}
			\begin{itemize}
				\item \texttt{VIEW\_ROOMS} - Chỉ xem
				\item \texttt{VIEW\_SERVICES} - Xem dịch vụ
				\item \texttt{PROVIDE\_SERVICES} - Cung cấp DV
				\item \texttt{VIEW\_DASHBOARD} - Xem tổng quan
			\end{itemize}
			
			\vspace{0.5em}
			\small\textit{Service không có quyền thêm/sửa/xóa, chỉ phục vụ khách}
		\end{column}
	\end{columns}
\end{frame}

% ===== ĐẶC TẢ VÀ SEQUENCE: ĐĂNG NHẬP =====
\begin{frame}{2.6 Đặc tả Use Case ``Đăng nhập''}
	\begin{block}{Use Case chung cho tất cả Actor}
		\begin{tabular}{|l|p{8.5cm}|}
			\hline
			\textbf{Tác nhân} & Tất cả (Manager, Staff, Service) \\
			\hline
			\textbf{Mô tả} & Xác thực người dùng để vào hệ thống \\
			\hline
			\textbf{Tiền điều kiện} & Có tài khoản trong hệ thống \\
			\hline
			\textbf{Luồng chính} & 1. Mở LoginFrame \\
			& 2. Nhập username và password \\
			& 3. AuthService kiểm tra thông tin \\
			& 4. Lưu User vào UserSession (Singleton) \\
			& 5. Chuyển sang MainFrame \\
			\hline
			\textbf{Luồng phụ} & Sai thông tin $\rightarrow$ Hiển thị lỗi, cho nhập lại \\
			\hline
			\textbf{Hậu điều kiện} & UserSession chứa User đang đăng nhập \\
			\hline
		\end{tabular}
	\end{block}
\end{frame}

\begin{frame}{2.7 Sequence Diagram - Đăng nhập}
	\begin{columns}[T]
		\begin{column}{0.6\textwidth}
			\begin{figure}
				\centering
				\includegraphics[width=\textwidth, height=0.6\textheight, keepaspectratio]{DangNhap.drawio.png}
				\caption{Biểu đồ tuần tự ``Đăng nhập''}
			\end{figure}
		\end{column}
		
		\begin{column}{0.4\textwidth}
			\textbf{Các bước thực hiện:}
			\begin{enumerate}
				\item User nhập username/password
				\item Nhấn nút Đăng nhập
				\item LoginFrame.performLogin()
				\item authenticate() kiểm tra credentials
				\item UserSession.login() lưu thông tin
				\item Mở MainFrame
			\end{enumerate}
			
			\vspace{0.5em}
			\textbf{Class liên quan:}
			\begin{itemize}
				\item \texttt{LoginFrame}
				\item \texttt{UserSession} (Singleton)
				\item \texttt{MainFrame}
			\end{itemize}
		\end{column}
	\end{columns}
\end{frame}

% ===== ĐẶC TẢ VÀ SEQUENCE: ĐẶT PHÒNG =====
\begin{frame}{2.8 Đặc tả Use Case ``Đặt phòng''}
	\begin{block}{Use Case quan trọng nhất - Nghiệp vụ cốt lõi của hệ thống}
		\begin{tabular}{|l|p{8.5cm}|}
			\hline
			\textbf{Tác nhân} & Lễ tân (Staff) \\
			\hline
			\textbf{Mô tả} & Tạo booking mới cho khách hàng, hệ thống tự động tính giá \\
			\hline
			\textbf{Tiền điều kiện} & Đã đăng nhập với quyền CREATE\_BOOKING \\
			\hline
			\textbf{Luồng chính} & 1. Lễ tân mở AddBookingDialog \\
			& 2. Chọn ngày check-in/check-out \\
			& 3. Hệ thống gọi \texttt{getAvailableRooms()} \\
			& 4. Chọn khách hàng từ CustomerManager \\
			& 5. Chọn phòng và xác nhận \\
			& 6. Hệ thống tạo Booking với status PENDING \\
			\hline
			\textbf{Luồng phụ} & Không có phòng trống $\rightarrow$ Thông báo lỗi \\
			\hline
			\textbf{Hậu điều kiện} & Booking được lưu, Room status = OCCUPIED \\
			\hline
		\end{tabular}
	\end{block}
\end{frame}

\begin{frame}{2.9 Sequence Diagram - Đặt phòng}
	\begin{columns}[T]
		\begin{column}{0.6\textwidth}
			\begin{figure}
				\centering
				\includegraphics[width=\textwidth, height=0.6\textheight, keepaspectratio]{DatPhong.drawio.png}
				\caption{Biểu đồ tuần tự ``Đặt phòng''}
			\end{figure}
		\end{column}
		
		\begin{column}{0.4\textwidth}
			\textbf{Luồng xử lý:}
			\begin{enumerate}
				\item Lễ tân chọn ``Thêm đặt phòng''
				\item openAddDialog() mở form
				\item Nhập ngày, refreshAvailableRooms()
				\item bookingManager.isRoomAvailable()
				\item Chọn Customer + Room
				\item saveBooking() $\rightarrow$ add(booking)
				\item DataStorage.saveAllData()
			\end{enumerate}
		\end{column}
	\end{columns}
\end{frame}

% ===== ĐẶC TẢ VÀ SEQUENCE: CHECK-OUT =====
\begin{frame}{2.10 Đặc tả Use Case ``Check-out và Tạo hóa đơn''}
	\begin{block}{Use Case kết hợp Check-out và Tạo hóa đơn}
		\begin{tabular}{|l|p{8.5cm}|}
			\hline
			\textbf{Tác nhân} & Lễ tân (Staff) \\
			\hline
			\textbf{Mô tả} & Hoàn tất booking và tạo hóa đơn thanh toán \\
			\hline
			\textbf{Tiền điều kiện} & Đã đăng nhập, Booking có status CHECKED\_IN \\
			\hline
			\textbf{Luồng chính} & 1. Lễ tân chọn booking cần check-out \\
			& 2. Xác nhận check-out \\
			& 3. Hệ thống đổi status = COMPLETED \\
			& 4. \texttt{<<include>>} Tạo hóa đơn tự động \\
			& 5. Tính tiền phòng + thuế \\
			& 6. Hiển thị Invoice \\
			\hline
			\textbf{Luồng phụ} & Booking không hợp lệ $\rightarrow$ Thông báo lỗi \\
			\hline
			\textbf{Hậu điều kiện} & Booking = COMPLETED, Invoice được tạo \\
			\hline
		\end{tabular}
	\end{block}
\end{frame}

\begin{frame}{2.11 Sequence Diagram - Check-out và Tạo hóa đơn}
	\begin{columns}[T]
		\begin{column}{0.6\textwidth}
			\begin{figure}
				\centering
				\includegraphics[width=\textwidth, height=0.6\textheight, keepaspectratio]{Check-outvaTaoHoaDon.drawio.png}
				\caption{Biểu đồ tuần tự ``Check-out và Tạo hóa đơn''}
			\end{figure}
		\end{column}
		
		\begin{column}{0.4\textwidth}
			\textbf{Luồng xử lý:}
			\begin{enumerate}
				\item Lễ tân chọn booking từ InvoicePanel
				\item getById() lấy thông tin booking
				\item Xác nhận tạo hóa đơn
				\item createInvoiceFromBooking()
				\item markAsPaid()
				\item saveAllData() lưu dữ liệu
			\end{enumerate}
		\end{column}
	\end{columns}
\end{frame}

% =================================================================
% PHẦN 3: THIẾT KẾ HỆ THỐNG
% =================================================================
\section{Thiết kế hệ thống}

\useStyleLogoOnly
\useThickBar

\begin{frame}{3.1 Kiến trúc MVC - 4 tầng}
	\begin{columns}[T]
		\begin{column}{0.5\textwidth}
			\begin{block}{Tại sao dùng kiến trúc MVC?}
				\begin{itemize}
					\item \textbf{Tách biệt} View - Controller - Model
					\item \textbf{Dễ bảo trì}: Sửa UI không ảnh hưởng logic
					\item \textbf{Dễ test}: Mock Storage để test Manager
				\end{itemize}
			\end{block}
			
			\vspace{0.5em}
			
			\textbf{4 Layer:}
			\begin{enumerate}
				\item \textbf{UI Layer} (Swing Panels)
				\item \textbf{Service Layer} (Managers)
				\item \textbf{Model Layer} (Entities)
				\item \textbf{Storage Layer} (JSON)
			\end{enumerate}
		\end{column}
		
		\begin{column}{0.5\textwidth}
			\begin{alertblock}{Luồng dữ liệu}
				\centering
				\texttt{UI Panel} \\
				$\downarrow$ \\
				\texttt{Manager (Business Logic)} \\
				$\downarrow$ \\
				\texttt{Storage (Serialize/Deserialize)} \\
				$\downarrow$ \\
				\texttt{JSON Files}
			\end{alertblock}
			
			\vspace{0.5em}
			\small
			\textit{Manager không biết dữ liệu lưu ở đâu → Có thể đổi sang Database sau!}
		\end{column}
	\end{columns}
\end{frame}

\begin{frame}{3.2 Class Diagram - Interfaces}
	\begin{columns}[T]
		\begin{column}{0.4\textwidth}
			\textbf{Tại sao dùng Interface?}
			\begin{itemize}
				\item \texttt{IManageable<T>}: CRUD chung
				\item \texttt{ISearchable<T>}: Tìm kiếm
				\item \texttt{IStorable<T>}: Lưu/tải
			\end{itemize}
			
			\vspace{0.5em}
			
			\textbf{Lợi ích (Abstraction):}
			\begin{itemize}
				\item Định nghĩa ``hợp đồng''
				\item Dễ mở rộng Manager mới
				\item Hỗ trợ Dependency Injection
			\end{itemize}
		\end{column}
		
		\begin{column}{0.6\textwidth}
			\begin{figure}
				\centering
				\includegraphics[width=0.95\textwidth, height=0.6\textheight, keepaspectratio]{Interfaces.drawio.png}
				\caption{Class Diagram - Interfaces}
			\end{figure}
		\end{column}
	\end{columns}
\end{frame}

\begin{frame}{3.3 Class Diagram - Room (Inheritance)}
	\begin{columns}[T]
		\begin{column}{0.4\textwidth}
			\textbf{OOP - Inheritance:}
			\begin{itemize}
				\item \texttt{Room} là abstract class
				\item 3 subclass: Standard, VIP, Deluxe
			\end{itemize}
			
			\vspace{0.5em}
			
			\textbf{OOP - Polymorphism:}
			\begin{itemize}
				\item Override \texttt{calculatePrice()}
				\item Standard: $\times 1.0$
				\item VIP: $\times 1.2$
				\item Deluxe: $\times 1.5$
			\end{itemize}
		\end{column}
		
		\begin{column}{0.6\textwidth}
			\begin{figure}
				\centering
				\includegraphics[width=\textwidth, height=0.65\textheight, keepaspectratio]{RoomClass.drawio.png}
				\caption{Abstract Room và 3 subclass}
			\end{figure}
		\end{column}
	\end{columns}
\end{frame}

\begin{frame}{3.4 Class Diagram - Entity (Composition)}
	\begin{columns}[T]
		\begin{column}{0.4\textwidth}
			\textbf{Quan hệ Composition:}
			\begin{itemize}
				\item Booking chứa Customer + Room
				\item Invoice chứa Booking
				\item Khi xóa Booking → Invoice bị ảnh hưởng
			\end{itemize}
			
			\vspace{0.5em}
			
			\textbf{OOP - Encapsulation:}
			\begin{itemize}
				\item Thuộc tính \texttt{private}
				\item Truy cập qua getter/setter
			\end{itemize}
		\end{column}
		
		\begin{column}{0.6\textwidth}
			\begin{figure}
				\centering
				\includegraphics[width=\textwidth, height=0.65\textheight, keepaspectratio]{Entity.drawio.png}
				\caption{Quan hệ Composition giữa Entity}
			\end{figure}
		\end{column}
	\end{columns}
\end{frame}

\begin{frame}{3.5 Class Diagram - Tổng thể}
	\begin{figure}
		\centering
		\includegraphics[width=0.9\textwidth, height=0.75\textheight, keepaspectratio]{ClassTongQuan.drawio.png}
		\caption{Class Diagram tổng thể hệ thống}
	\end{figure}
\end{frame}

\begin{frame}{3.6 Design Patterns sử dụng}
	\begin{columns}[T]
		\begin{column}{0.33\textwidth}
			\begin{block}{Factory Pattern}
				\texttt{RoomFactory}
				\begin{itemize}
					\item Tạo Room theo type
					\item Ẩn logic khởi tạo
					\item Dễ thêm loại mới
				\end{itemize}
			\end{block}
		\end{column}
		
		\begin{column}{0.33\textwidth}
			\begin{block}{Singleton Pattern}
				\texttt{UserSession}
				\begin{itemize}
					\item 1 instance duy nhất
					\item Lưu thông tin login
					\item Truy cập toàn cục
				\end{itemize}
			\end{block}
		\end{column}
		
		\begin{column}{0.33\textwidth}
			\begin{block}{MVC Pattern}
				Toàn bộ kiến trúc
				\begin{itemize}
					\item View: UI Panels
					\item Controller: Managers
					\item Model: Entities
				\end{itemize}
			\end{block}
		\end{column}
	\end{columns}
	
	\vspace{0.5em}
	
	\centering
	\textbf{Tại sao dùng Design Patterns?} $\rightarrow$ Code dễ đọc và bảo trì
\end{frame}

% =================================================================
% PHẦN 4: TRIỂN KHAI
% =================================================================
\section{Triển khai}

\useStyleHustFull
\useThinBar

\begin{frame}{4.1 Tổ chức mã nguồn}
	\begin{columns}[T]
		\begin{column}{0.5\textwidth}
			\textbf{Cấu trúc thư mục:}
			\begin{itemize}
				\item \texttt{src/com/hotel/}
				\begin{itemize}
					\item \texttt{model/} - 11 files
					\item \texttt{service/} - 7 files
					\item \texttt{storage/} - 2 files
					\item \texttt{ui/} - 17 files
					\item \texttt{auth/} - 2 files
					\item \texttt{util/} - 2 files
				\end{itemize}
			\end{itemize}
			
			\vspace{0.5em}
			
			\textbf{Tổng cộng:} $\sim$41 files, $\sim$4500 LOC
		\end{column}
		
		\begin{column}{0.5\textwidth}
			\begin{block}{Data Files (JSON)}
				\begin{itemize}
					\item \texttt{rooms.json} - Danh sách phòng
					\item \texttt{customers.json} - Khách hàng
					\item \texttt{bookings.json} - Đặt phòng
					\item \texttt{invoices.json} - Hóa đơn
					\item \texttt{users.json} - Tài khoản
				\end{itemize}
			\end{block}
			
			\vspace{0.5em}
			
			\textbf{Tại sao dùng JSON?}
			\begin{itemize}
				\item Đơn giản, không cần DB server
				\item Human-readable, dễ debug
				\item Portable, đi kèm project
			\end{itemize}
		\end{column}
	\end{columns}
\end{frame}

\begin{frame}{4.2 Bốn nguyên lý OOP áp dụng}
	\begin{columns}[T]
		\begin{column}{0.5\textwidth}
			\begin{block}{1. Encapsulation (Đóng gói)}
				\begin{itemize}
					\item Thuộc tính \texttt{private}
					\item Truy cập qua getter/setter
					\item Ví dụ: \texttt{Room.basePrice}
				\end{itemize}
			\end{block}
			
			\vspace{0.5em}
			
			\begin{block}{2. Inheritance (Kế thừa)}
				\begin{itemize}
					\item \texttt{Room} → \texttt{StandardRoom}
					\item \texttt{Room} → \texttt{VIPRoom}
					\item \texttt{Room} → \texttt{DeluxeRoom}
				\end{itemize}
			\end{block}
		\end{column}
		
		\begin{column}{0.5\textwidth}
			\begin{block}{3. Polymorphism (Đa hình)}
				\begin{itemize}
					\item \texttt{calculatePrice()} trả về giá khác nhau
					\item VIPRoom: $\times 1.2$
					\item DeluxeRoom: $\times 1.5$
				\end{itemize}
			\end{block}
			
			\vspace{0.5em}
			
			\begin{block}{4. Abstraction (Trừu tượng)}
				\begin{itemize}
					\item Abstract class: \texttt{Room}
					\item Interface: \texttt{IManageable<T>}
					\item Interface: \texttt{ISearchable<T>}
				\end{itemize}
			\end{block}
		\end{column}
	\end{columns}
\end{frame}

% =================================================================
% PHẦN 5: KẾT LUẬN
% =================================================================
\section{Kết luận}

\begin{frame}{5.1 Tổng kết}
	\begin{columns}[T]
		\begin{column}{0.5\textwidth}
			\textbf{Đã hoàn thành:}
			\begin{itemize}
				\item[$\checkmark$] Hệ thống CRUD hoàn chỉnh
				\item[$\checkmark$] 4 tính chất OOP
				\item[$\checkmark$] 3 Design Patterns
				\item[$\checkmark$] Kiến trúc MVC 4 tầng
				\item[$\checkmark$] Phân quyền Permission-based
				\item[$\checkmark$] UI hiện đại (FlatLaf)
			\end{itemize}
		\end{column}
		
		\begin{column}{0.5\textwidth}
			\textbf{Hướng phát triển:}
			\begin{itemize}
				\item Chuyển sang Database thực
				\item Thêm module Dịch vụ phòng
				\item Tích hợp thanh toán online
				\item Responsive Web UI
			\end{itemize}
			
			\vspace{1em}
			
			\begin{alertblock}{Demo}
				\centering
				\textbf{DEMO ỨNG DỤNG}
			\end{alertblock}
		\end{column}
	\end{columns}
\end{frame}

% =================================================================
% LIÊN HỆ & CẢM ƠN
% =================================================================

\hustcontactpage{\hfill THÔNG TIN LIÊN HỆ \hfill }{
	\textbf{Nhóm 204:} \\
	- Trần Đình Khánh - 20237449 \\
	- Nguyễn Hữu Linh - 20237455 \\
	\vspace{0.5cm}
	
	\begin{itemize}
		\item \textbf{GitHub:} \url{https://github.com/IvanLyVodka11/hotel-management}
	\end{itemize}
	
	\vspace{1cm}
	\textit{Xin chân thành cảm ơn Thầy/Cô và các bạn đã lắng nghe!}
}

% Trang Q&A
\useStyleLogoOnly
\setbeamertemplate{bibliography item}[text]

\begin{frame}{Q\&A}
	\centering
	\vspace{2cm}
	
	{\Huge \textbf{CẢM ƠN QUÝ THẦY CÔ!}}
	
	\vspace{2cm}
	
	{\Large Mời Thầy/Cô đặt câu hỏi}
\end{frame}

\hustthankyou

\end{document}